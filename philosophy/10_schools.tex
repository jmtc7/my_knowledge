\documentclass[../my_knowledge.tex]{subfiles}

\begin{document}

\section{10 Main Schools of Philosophy}
The content of this subsection was extracted from BigThink's \href{https://bigthink.com/thinking/10-schools-of-philosophy-and-why-you-should-know-them/}{\textit{10 schools of philosophy and why you should know them}} \cite{10schools}

\subsection{Nihilism}
It consist of a group of view angles and problems rather than being an actual school of thought. It's approach is to question the existance of meaning in something. Some examples would be:

\begin{itemize}
  \item Moral Nihilism, which proposes that nothing can be affirmed to be \textit{moral}.
  \item Metaphysical Nihilism says that no metaphysical affirmation can be a facts.
  \item Existential Nihilism considers that life itself does not possess any meaning or value. This problem is what is most commonly related to the Nihilism.
\end{itemize}

As opposed to popular belief, Nietzsche was not a nihilist. Rather, he wrote about the dangers posed by nihilism and offered solutions to them. Real nihilists included the
\href{https://en.wikipedia.org/wiki/Russian_nihilist_movement}{Russian Nihilist movement}.

\subsection{Existentialism}
Originated in the work of \href{https://www.youtube.com/watch?v=D9JCwkx558o}{Soren Kierkegaard} and \href{https://www.youtube.com/watch?v=wHWbZmg2hzU}{Nietzsche}, existentialism focuses on the problems posed by existential nihilism. It dives in the problematics that a nihilist approach of our existance brings. Existentialists would philosophize about the fact that, if there is no particular purpose on life, there would not be any point on living. These thoughts lead to question many things, such as our ability to act freely. This approach would also bring many difficulties to exist outside of a system, as an individual.

The existentialists also included \href{https://www.youtube.com/watch?v=3bQsZxDQgzU}{Jean-Paul Sartre}, \href{https://bigthink.com/the-present/ten-women-of-philosophy-and-why-you-should-know-them/}{Simone de Beauvoir}, and \href{https://www.youtube.com/watch?v=Br1sGrA7XTU&t=2s}{Martin Heidegger}. \href{https://www.youtube.com/watch?v=jQOfbObFOCw}{Albert Camus} was associated with the movement, but considered himself independent of it.

\subsection{Stoicism}
A philosophy popular in ancient Greece and Rome, and practiced today by many people in high-stress environments, such as \href{https://www.si.com/nfl/2015/12/08/ryan-holiday-nfl-stoicism-book-pete-carroll-bill-belichick}{athletes}.

Stoicism is a school that focuses on existance in a non-ideal environment. Whenever something undesirable happens around one self, the way to go is acceptance. Pain and discomfort will pass, but we will remain, so we should not let things beyond our control distract or disrupt us but. Rather, we shall focus in what we are actually able to control.

Famed stoics included Zeno of Citium, Seneca, and Marcus Aurelius.

\subsection{Hedonism}
Hedonism affirms that pleasure/happiness is the only thing that matters, the only thing that provide value to us. Other schools, such as the utilitarians, defended the same idea. Often, Hedonism is related to the idea of accepting depravity and the absence of limits. However, some hedonists thinkers, such as Epicurus, linked happiness and pleasure with a moderation-based moral and ethics.

Many hedonistic philosophers considered pleasure to be a kind of happiness, but not the only one. Often, different happiness types would be ranked and actions that increase your culture level were comonly considered to bring a higher happiness than excesses in other actions or the blind following of an instinct, which are the things commonly meant when someone uses the word \textit{hedonistic}, when used as a slur.

Famous hedonists include Jeremy Bentham, \href{https://www.youtube.com/watch?v=Kg_47J6sy3A}{Epicurus}, and Michel Onfray. Hedonism is also the oldest philosophy recorded, making an appearance in \textit{The Epic of Gilgamesh}.

\subsection{Marxism}
\href{https://www.youtube.com/watch?v=fSQgCy_iIcc}{Karl Marx} is the father of this school, and it developed after his death with the contributions of others. The core of it is a criticism of the capitalism, arguing that it makes us not being aware of the results of our work and that it tends to produce more than needed and, as a consequence, it suffers cyclical crashes. It also proposes that \textit{value} comes from the work and effort invested in doing something. This is, it says that when something is sold, it has a certain value because people worked to generate it. He also proposed some ideas to fix, or reduce, some of the issues that he found in the capitalist approach.

This school shall not be mistaken by the \href{https://en.wikipedia.org/wiki/Cultural_Marxism_conspiracy_theory}{conspiracy theory of the \textit{Cultural Marxism}}, which claims that a big part of the inestability of the interwar period is planned by the marxists and the \href{https://en.wikipedia.org/wiki/Frankfurt_School}{Frankfurt School} to subvert the western society. Some say this school was actually oriented to the criticism of consumism and our tendency to commodity, while others defend that it had a strong political side that rejected the ways of the western capitalist approach to government.

Famed Marxists include Lenin, Stalin, Mao, and \href{https://bigthink.com/experts/slavoj-zizek}{Slavoj Zizek}; though all of the listed individuals have been called heretics at one point or another by other Marxists. Ironically, Marx himself claimed to not be one.

\subsection{Logical Posistivism}
This school tries to use logic and empirical proofs as the base for everything.

Particularly popular in the early 20\textsuperscript{th} century, it proposed that many philosophical topics, such as metaphysics, ethics, theology or aesthetics could not be studied due to their lack of true value. However, this same belief could be reciprocally applied to itself, making a paradox out of this school, since it is always possible to doubt about the veracity of a verification. With these kind of thoughts, the school and the \href{https://en.wikipedia.org/wiki/Verificationism}{verificationism} arrived to a decline that ended up killing it.

\href{https://www.youtube.com/watch?v=pQ33gAyhg2c}{Ludwig Wittgenstein} caused great damage to this school when he rejected his previous work in favor of it. However, this line of though had a considerable influence in future thinkers, amongst which Karl Popper and Wittgenstein are to be highlighted for their work to disprove core tenets.

Famed members of the movement included Bertrand Russell, Ludwig Wittgenstein, and the Vienna Circle.

\subsection{Taoism}
\href{https://www.youtube.com/watch?v=dFb7Hxva5rg}{Taoism} is based in the \textit{Tao Te Ching}, written by the ancient Chinese philosopher Lao-Tzu when he quitted China to pursue the life of a hermit. This school proposes that we should focus in ourselves, following the \textit{Way}, which is based in valueing the simple and natural things. It is practiced as a folk religion by the Chinese, and Taoists may make offerings to various gods.

Taoism has common elements with Buddhism and Zen, and parts of it were taken by the Neo-Confucianism. The way this school Viewing opposites as complementary is one of Taoism's principles, which influenced people such as the physicist Niels Bohr. 

\subsection{Rationalism}
It is based in the fact that our senses may misinterpretate things, so this school would not trust in their ability to provide us with a true perception of reality. Thus, sice the senses can not be trusted, the only way we have to perceive reality is the reasoning, not empirical evidence.

This was a popular idea trough history, with defenders such as Socrates, Rene Descartes and \href{https://www.youtube.com/watch?v=pVEeXjPiw54}{Spinoza}. However, the idea that only the reason can be trusted when trying to reveal the actual nature of the world has been losing popularity, favouring a combination between rationalist basis and empirical observations.

\subsection{Relativism}
The relativist approach is based in the idea that things are relative to the perspective with which we look at them. It therefore questions the existance of an absolute moral or truth.

\textit{Cultural Relativism} argues that it is not possible to compare the morality of two cultures and no one can criticize anotherone's culture because that would mean, that other person could do the exact same thing. However, this is often not strongly supported and it is usually considered to be self-defeating by the people philosophizing about ethics. 

\subsection{Buddhism}
It is a religion based around the teachings of \href{https://www.youtube.com/watch?v=tilBs32zN7I}{Siddhartha Gautama} (Buddha), an Indian prince. Buddhism's tenet is that suffering has a cause and it can be overcomed by meditation, by following the \href{https://en.wikipedia.org/wiki/Noble_Eightfold_Path}{\textit{Nobel Eightfold Path}}, and by contemplating \textit{sutras}.

While there are many buddhist schools, all are connected by their belief in Buddha's vision on suffering. Some of their differences come from their divine figures, that can go from not having any god to having a pantheon with several divine and demonic figures. Also, they may differ in the existance of \textit{karma} and reincarnation, or even afterlife itself.

The many schools of Buddhism are rather diverse in their thought, bound together primarily by Buddha’s ideas on suffering and meditattion. However, some are non-theistic while others have a pantheon of gods and demons. Some hold that karma exists and reincarnation is a part of life while others reject any discussion of an afterlife.

\end{document}