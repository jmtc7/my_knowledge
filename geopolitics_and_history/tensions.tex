\documentclass[../my_knowledge.tex]{subfiles}

\begin{document}

\section{Geopolitical Tensions}

\subsection{Spanish Separatist Movements}
\subsubsection{Basque Nationalism}
TODO

\subsubsection{Catalan Nationalism}
TODO

\subsection{Hispano-Moroccan Tensions}
The content of this subsection was mostly extracted from \href{https://www.youtube.com/watch?v=5ADhiE_tES0}{Memorias de Pez}\cite{conflict_morocco_spain}.

Conflicts between Marocco and Spain originate mostly from one of the following topics:

\begin{itemize}
	\item \textbf{Territorial disputes} regarding the pieces of land owned by Spain very close to the Maroccan coast, or even in mainland Africa with a direct border with Morocco, such as Ceuta, Melilla and Pe\~non de V'elez de la Gomera (refer to \ref{hispano_moroccan_border} for more details concerning this topic).
	\item \textbf{Fishing}: Spain does not have enough water around to use all its fishing potential, specially in the South. This is why, with the help of the European Union, big amounts of money is payed to Morocco so that Spanish fishermen can use Moroccan waters. However, Morocco considers that Spain/the EU should also pay for the water that corresponds to Western Sahara, since Morocco considers that area part of its territory. That was agreed by the EU after some polemic negotiations with people from that area. This agreement finishes in 2023.
	\item \textbf{Western Sahara}: Spain owned this area of Africa, but it was left in 1975. After that, Morocco occupied part of Western Sahara and claims that he owns it all, which led to armed conflicts between Morocco and the Polisario Front, a independentist group from Western Sahara. Morocco is sending more and more people to occupy Western Sahara lands, action that, by 2021, allowed Morocco to reach a 50\% occupation of the area. Since 1991 there was a peace treaty, but it was broke in 2020. The United Nations gives Spain the responsability to solve the conflict, since it is the only recognized authority over these lands. However, no Spanish government supported any sides until 2022, after the war between Russia and Ukraine started. Spain supported Morocco, so that they would open the pipelines that would allow Spain to access Algerian gas. However, during year, the passive attidue of Spain regarding this topic disturbed Morocco.
	\item \textbf{Illegal immigration} in the Hispano-Maroccan border has been a problem since many years. Sub-Saharan Africa and Maroccan people try to find their ways into Spain trough its borders with Morocco, mostly in Ceuta, Melilla and Canary Islands. The way of solving this of both Spain and the European Union is to pay Morocco (with money and political support) so that they would take care of the borders. This has been a polemic subject for a while, since some people argue that Morocco uses unethical practices to accomplish this mission, and that such actions are allowed by Spain and the EU. Also, Morocco uses this \textit{responsability} as a political weapon, being more or less strict in the immigration control to put or remove pressure in Spain/UE when needed, something similar to what Turkey does. 
\end{itemize}

One of the most recent tension points (that ended up in immigration pressure) between these two countries happened the 18\textsuperscript{th} April 2021, when a medical plane flew from Argel (Algeria) to Logro\~no (Spain), where Brahim Gali was travelling for a health issue. This man was the General Secretary of the Polisario Front and the President of the Democratic Sahrawi Arab Republic, the govenment of the Western Sahara that Morocco does not recognize. Also, the political situation of Spain, with many political parties with very different positions regarding immigration plays in favour of Morocco, making it easier for it to put pressure on Spain using illegal immigration.

\end{document}