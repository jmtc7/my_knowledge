\documentclass[../my_knowledge.tex]{subfiles}

\begin{document}

\section{Borders}
Borders define the limits of continents, countries or regions. Some are defined by historical facts, others by cultural differences, but they are mostly based in political (or geopolitical) decisions, either taken by groups of population that decides how to split the land they inhabit or by third parties that, one way or another, happened to have the power of define those borders.

\subsection{Spanish Borders}
This subsection is mostly inspired by \href{https://www.youtube.com/watch?v=L6tJ-mvhznU}{RealLifeLore}\cite{spanish_borders}.

While Spain is often seen as just a regular peninsula in the westernmost Europe, it also possess several islands, and some very particular borders with the nearby countries.

\subsubsection{Hispano-Moroccan Border}
\label{hispano_moroccan_border}
After the \textit{Reconquista} (801 - 1492 A.D.), the Spanish Kingdom conquered several North-African territories, some of which still remain Spanish. These pieces of land are claimed by Morocco until this day. Most of these territories are islands near the African coast. However, three are actually part of mainland Africa, those being:

\begin{itemize}
	\item \textbf{Pe\~n\'on de V\'elez de la Gomera}: It is located around 100 Km towards the East from the Strait of Gibraltar. It consists in two rocks, a big one (hosting a small town) and a smaller one. It is separated from morocco by an 80 m wide sand beach, making it the shortest international border segment in the world (given than the actual border with Morocco, when adding up all the Spanish territories, is bigger than the border between Zambia and Botswana\cite{shortest_borders})
	\item \textbf{Ceuta and Melilla}: These two autonomous cities, with almost 80,000 people living in each of them, are surrounded by 6 m high fences and are the only European Union territories in Africa. Unlike Melilla, Ceuta does not have an airport, so a boat is needed to get to mainland Spain. On top of that, a single road next to the East coast connects Ceuta with Morocco, making it quite isolated.
\end{itemize}

Regarding the islands, the most resonated case is the one of the \textbf{Isla Perejil}. It is a very small uninhabited rock near Ceuta. It is both claimed by Spain and Morocco, which leaded to an interesting event in 2002, when 12 Moroccan soldiers planted a Moroccan flag in it. It was an international conflict in which the Arab League backed up Morocco, while Spain was supported by NATO and the European Union. Several warships surrounded the tiny island and it all ended with an Spanish operation that closed the conflict arresting the Moroccan soldiers. Afterwards, the island was declared no-man zone with the agreement of both countries.

\subsubsection{Hispano-British Border}
This border separates Spain from a small British city, \textbf{Gibraltar}, located at the very South of the Iberian Peninsula. After the War of the Spanish Succession, in 1713, Spain gave it to the United Kingdom as part of the peace treaty. It has been a subject of dispute between the two contries.

\subsubsection{Hispano-French Border}
Last but not least, there are some pretty particular borders between Spain and France. One consists in the \textbf{Pheasant Island}, a small uninhabited islands in the middle of the Bidasoa River, at the West side of the border between Spain and France. In 1659, as part of a peace treaty, the sovereignty of this island was decided to be shared amongts France and Spain. Although no body is allowed in the island, the island belongs to either one of the abovementioned countries for only six months of each year, and there is ceremony to transfer such sovereignty each six months.

Another consequence of the same treaty is \textbf{Llivia}, a small Spanish city not far from the East border of Andorra completely surrounded by France and 1.6 Km away from the northern Spanish border. That French territory that is now surrounding Llivia was before a part of Spain, and it was transfered to France. Both countries, however, agreed that Llivia could remain being a part of Spain due to its importance at the time. This kind of situation is called an \textit{enclave}.

\end{document}